
%% ------------------------

\documentclass[12pt, a4paper]{article}

\usepackage{bm} % abych mohl psat tucne
\usepackage[czech]{babel} % cestina
\usepackage[T1]{fontenc}
\usepackage[utf8x]{inputenc}
\usepackage[unicode]{hyperref} % odkazy v pdf, www a na e-mail
\usepackage{graphicx} % obrazky
\usepackage{epstopdf} % obrazky
\usepackage{caption} % obrazky abych mohl udelat v popisku \\ ~
\usepackage{subcaption} % obrazky abych mohl udelat v popisku \\ ~
\usepackage{abraces, mathtools} % svorka zespoda
\linespread{1.10} % radkovani 1.3 odpovida radkovani 2
\usepackage{lmodern} % daji se pouzit \HUGE atd.

\usepackage[numbered,framed]{matlab-prettifier} %vkladani kodu matlabu
\lstset{style = Matlab-editor, basicstyle = \mlttfamily, escapechar = ", mlshowsectionrules = true,}

\usepackage{amsmath}
\usepackage{algorithm}
\usepackage[noend]{algpseudocode} % Vkladani pseudo kodu

\usepackage{color}




%------------------DRAFT----------------------------
\usepackage{draftwatermark}
\SetWatermarkLightness{0.95}
\SetWatermarkScale{1}
\SetWatermarkText{DRAFT}

%------------------LAYOUT----------------------------
\usepackage[top = 2.5 cm, bottom = 2.5 cm, left = 2.5 cm, right = 2.5 cm]{geometry} % geometrie stranky
\usepackage{longtable}% Pro dlouhy obsah, da se zalomit \pagebrek
\usepackage{fancyhdr}
\pagestyle{fancy}% Deffaultni nastaveni hlavicky a paticky
\setlength{\headheight}{16 pt}% Zvetsi hlavicku, aby to nedelalo warningy
\fancyhf{}
%\lhead{\bfseries \leftmark}
\fancyfoot[R]{\thepage}
\fancyfoot[L]{Poslední úpravy: \today}

%****************************************************
\begin{document}

 
%--------TITULNI-STRANA------------------------------
\begin{titlepage}
\begin{center}
	\vspace*{\fill}
	\textbf{\Huge{Obsah cvičení předmětu MPV}}\\
	\vspace*{\fill}
	\textbf{\large{\href{mailto:LBures@kky.zcu.cz}{Ing. Lukáš Bureš}}} \hfill \textbf{\large{Plzeň}}\\
	\textbf{\large{Katedra kybernetiky}} \hfill \textbf{\large{\today}}
\end{center}
\end{titlepage}


%--------OBSAH----------------------------------------
\setcounter{page}{1}
\pagenumbering{Roman}

\tableofcontents
\newpage

\setcounter{page}{1}
\pagenumbering{arabic}

%--------1----------------------------------------
\section{Cvičení}
\begin{itemize}
	\item \par{ \textbf{Obsah přednášky:} Adaptive Histogram Equalization, Contrast-Limited Adaptive Histogram Equaliyation (CLAHE), Vyhlzování histogramů, Non-Maximum Supression - automatický odhad prahu, Otsu metoda automatické detekce prahu, GMM - EM}
	\item \par{\textbf{Obsah cvičení:} Představení sebe, e-mail, kancelář, bude to těžší než ZDO, za co bude zápočet: 3 semestrální práce, v ideálním případě na sebe budou navazovat a budou se odevzdávat pomocí automatického odevzdávače (\textbf{SAKO} - \textbf{S}ystém \textbf{A}utomatické \textbf{KO}ntroly semestrálních prací), povolené programovací jazyky Python (Možná plus OpenCV) a MATLAB, na zápočet alespoň 60\% všech bodů, bude 1/4 bodů za aktivitu ve cvičeních?, jaká bude zkouška (kombinovaná? bude test? nebo jen ústní)}
	\par{Vyzkoušet si automatický odevzdávač (SAKO) jestli bude už k dispozici. Otevřít si MATLAB/Python a vyzkoušet si CLAHE, Otsu prahování + počítání objektů zrnka rýže atd. \href{http://docs.opencv.org/master/doc/py_tutorials/py_imgproc/py_histograms/py_histogram_equalization/py_histogram_equalization.html}{(Histogram Equalization, CLAHE, OpenCV, Python)}}
	
\end{itemize}
\noindent\rule[0.5ex]{\linewidth}{0.4pt}










%--------2----------------------------------------
\section{Cvičení}
\begin{itemize}
	\item \par{ \textbf{Obsah přednášky:} Mean-shift - princip, ukázka segmentace podle barev, Markov Random Fields (MRF), Graph-Cut}
	\item \par{\textbf{Obsah cvičení:} Max-flow vysvětlení na 2px příkladu, Max-flow naprogramovat, praktická ukázka segmentace, naprogramovat Mean-shift, \href{http://docs.opencv.org/trunk/doc/py_tutorials/py_imgproc/py_grabcut/py_grabcut.html}{Grab-Cut {\color{red}???}} \href{http://docs.opencv.org/trunk/doc/py_tutorials/py_video/py_meanshift/py_meanshift.html}{Meanshift and Camshift, OpenCV, Python}, \href{http://scikit-learn.org/stable/auto_examples/cluster/plot_cluster_comparison.html}{sklearn mean-shift}, \href{http://research.microsoft.com/pubs/150437/ibfs-proc.pdf}{Max-flow pro huste grafy - rychlejsi}}
\end{itemize}
\noindent\rule[0.5ex]{\linewidth}{0.4pt}











%--------3----------------------------------------
\section{Cvičení}
\begin{itemize}
	\item \par{\textbf{Obsah přednášky:} Spline + Active Contours, Canny Corner Detector, Harris Corner Detector}
	\item \par{\textbf{Obsah cvičení:} Popsat spline zmínit NURBS, naprogramovat Harris Corner Detector (bude připravená kostra programu), dále naprogramovat interpolaci získaných bodů pomocí jednoduché (kubické) interpolace , vyzkoušet SAKO podruhé, \href{http://docs.opencv.org/master/doc/py_tutorials/py_feature2d/py_features_harris/py_features_harris.html}{Harris Corner Detector 1, OpenCV, Python}, \href{http://docs.opencv.org/doc/tutorials/features2d/trackingmotion/harris_detector/harris_detector.html}{Harris Corner Detector 2, OpenCV, C++}}, \href{http://www.janeriksolem.net/2009/01/harris-corner-detector-in-python.html}{Harris Corner Detector, Python, NO\_OpenCV}
	\item \par{\textbf{Zadání 1. semestrální práce:} {\color{red} ???}}
\end{itemize}
\noindent\rule[0.5ex]{\linewidth}{0.4pt}











%--------4----------------------------------------
\section{Cvičení}
\begin{itemize}
	\item \par{\textbf{Obsah přednášky:} SIFT, SURF, ORB, MSER, RANSAC}
	\item \par{\textbf{Obsah cvičení:} \href{http://docs.opencv.org/trunk/doc/py_tutorials/py_feature2d/py_sift_intro/py_sift_intro.html?highlight=sift}{SIFT}, \href{http://docs.opencv.org/trunk/doc/py_tutorials/py_feature2d/py_surf_intro/py_surf_intro.html?highlight=surf}{SURF}, \href{http://docs.opencv.org/trunk/doc/py_tutorials/py_feature2d/py_orb/py_orb.html}{ORB} použít, vykreslit, porovnat rychlost. MSER naprogramovat {\color{red}jak je hledat korespondence pro RANSAC ???} , RANSAC - matchování bodů, \href{http://docs.opencv.org/trunk/doc/py_tutorials/py_feature2d/py_sift_intro/py_sift_intro.html}{SIFT 1, OpenCV, Python}, \href{http://opencv-python-tutroals.readthedocs.org/en/latest/py_tutorials/py_feature2d/py_matcher/py_matcher.html}{SIFT 2, OpenCV, Python}}
\end{itemize}
\noindent\rule[0.5ex]{\linewidth}{0.4pt}












%--------5----------------------------------------
\section{Cvičení}
\begin{itemize}
	\item \par{\textbf{Obsah přednášky:} PCA, LDA, Active Shape Model (ASM), Active Appearance Model (AAM)}
	\item \par{\textbf{Obsah cvičení:} Naprogramovat Active Shape Model (ASM) ({\color{red}Míra Hlaváč dodá podklady}). PCA naprogramovat. \href{http://cs.wikipedia.org/wiki/Analýza_hlavních_komponent}{PCA, Wikkipedia 1}, \href{http://en.wikipedia.org/wiki/Principal_component_analysis}{PCA, Wikipedia EN 2}}
\end{itemize}
\noindent\rule[0.5ex]{\linewidth}{0.4pt}











%--------6----------------------------------------
\section{Cvičení}
\begin{itemize}
	\item \par{\textbf{Obsah přednášky:} Texturový popis, Hu + Zernikeho momenty, LBP, HoG, Gabor, Wavelet, Haar-like features}
	\item \par{\textbf{Obsah cvičení:} Face detector v OpenCV ukázat. Naprogramovat LBP, dotaz do DB textur, dej mi 9 nejbližších + vizualizace (bude napsáno).}
	\item \par{\textbf{Zadání 2. semestrální práce:} {\color{red}???}}
\end{itemize}
\noindent\rule[0.5ex]{\linewidth}{0.4pt}











%--------7----------------------------------------
\section{Cvičení}
\begin{itemize}
	\item \par{\textbf{Obsah přednášky:} Princip klasifikace, lineární regrese, binární regrese, regularizace, AdaBoost, SVM, SVM - kernel trick}
	\item \par{\textbf{Obsah cvičení:} Vysvětlení trénovací, testovací a cross validation sady. Precision, recall, F1 skore, false positiv atd. {\color{red}Bude dodán klasifikátor značka ne-značka od Marka} - úkolem bude z Google street-view si najít značku, printscreen, uložit jako obrázek a naprogramovat změnu velikosti spočítat HoG příznaky a klasifikovat jestli to je a nebo není značka.}
\end{itemize}
\noindent\rule[0.5ex]{\linewidth}{0.4pt}













%--------8----------------------------------------
\section{Cvičení}
\begin{itemize}
	\item \par{\textbf{Obsah přednášky:} Rozhodovací strom, rozhodovací les, náhodný rozhodovací les, Bayes klasifikátor, neuronové sítě - základ, plus ukázka hlubokých neuronových sítí}
	\item \par{\textbf{Obsah cvičení:} {\color{red} Naprogramovat náhodný rozhodovací strom/les?} Porovnání klasifikátorů pomocí \textit{sklearn} (\href{http://scikit-learn.org/stable/auto_examples/plot_classifier_comparison.html}{porovnání klasifikátorů})}
\end{itemize}
\noindent\rule[0.5ex]{\linewidth}{0.4pt}
















%--------9----------------------------------------
\section{Cvičení}
\begin{itemize}
	\item \par{\textbf{Obsah přednášky:} Optický tok, Lukas-Kanade, Background Subtraction, Block-matching, CAM-Shift.}
	\item \par{\textbf{Obsah cvičení:} Cam-Shift v malování si vytvořit objekt, který se bude sledovat (objekt barevný okolí bílé / černé). Naprogramovat Background Subtraction (BS) natočených videích kutálejícího se míčku do krabice (porovnávat histogramy bloků obrazu ve dvou po sobě jdoucích snímcích). Ukázka OpenCV optického toku. \href{http://docs.opencv.org/trunk/doc/py_tutorials/py_video/py_lucas_kanade/py_lucas_kanade.html}{Optical Flow, Lucas-Kanade, OpenCV, Python}}
	\item \par{\textbf{Zadání 3. semestrální práce:}}
\end{itemize}
\noindent\rule[0.5ex]{\linewidth}{0.4pt}













%-------10----------------------------------------
\section{Cvičení}
\begin{itemize}
	\item \par{\textbf{Obsah přednášky:}Kalman Filter, Particle filter}
	\item \par{\textbf{Obsah cvičení:} {\color{red} Particle Filter Marek pomůže s implementací.} Jinak bude Kalmanův Filtr s touto ukázkou \href{http://www.morethantechnical.com/2011/06/17/simple-kalman-filter-for-tracking-using-opencv-2-2-w-code/}{Kalman Filter, Odhad pohybu, OpenCV}, \href{http://www.hdm-stuttgart.de/~maucher/Python/ComputerVision/html/Tracking.html}{Tracking, Background Subtraction, OpenCV, Python}, , \href{http://jayrambhia.wordpress.com/2012/07/26/kalman-filter/}{Kalman Filter}}
\end{itemize}
\noindent\rule[0.5ex]{\linewidth}{0.4pt}













%-------11----------------------------------------
\section{Cvičení}
\begin{itemize}
	\item \par{\textbf{Obsah přednášky:} Definice geometrie, vnitřní a vnější parametry, fundamentální matice, epipolární geometrie, kalibrace kamery}
	\item \par{\textbf{Obsah cvičení:} Kalibrace jedné kamery, kalibrační vzor, měření reálné metriky, měření vzdálenosti předmětu od kamery vzhledem k měnící se šířce objektu. \href{http://opencv-python-tutroals.readthedocs.org/en/latest/py_tutorials/py_calib3d/py_calibration/py_calibration.html}{Kalibrace jedné kamery, OpenCV, Python}, \href{http://docs.opencv.org/doc/tutorials/calib3d/camera_calibration/camera_calibration.html}{Kalibrace, OpenCV, C++}}
\end{itemize}
\noindent\rule[0.5ex]{\linewidth}{0.4pt}







\newpage





%-------12----------------------------------------
\section{Cvičení}
\begin{itemize}
	\item \par{\textbf{Obsah přednášky:} Rektifikace obrazu, disparita, hloubková mapa, Kinect}
	\item \par{\textbf{Obsah cvičení:} Stereo snímky, stereo kalibrace, vnitřní parametry, fundamentální matice, vizualizace epipolár, rektifikace, {\color{red}3D rekonstrukce nalezených bodů je nutné domyslet.}}
	\par{{\color{blue} Zápočty pro ty, kdo mají už nárok na zápočet a odevzdány všechny semestrální práce.}}
\end{itemize}
\noindent\rule[0.5ex]{\linewidth}{0.4pt}













%-------13----------------------------------------
\section{Cvičení}
\begin{itemize}
	\item \par{\textbf{DEADLINE na odevzdávání prací!}}
	\item \par{\textbf{Obsah přednášky:} {\color{red}Předtermín ?}}
	\item \par{\textbf{Obsah cvičení:} {\color{blue} Zápočty pro ty co nechtějí jít na předtermín ?}}
\end{itemize}
\noindent\rule[0.5ex]{\linewidth}{0.4pt}














%-------SEMESTRALKY-------------------------------
\section*{Semestrální práce}
\begin{enumerate}
	\item \par{\textbf{Semestrální práce:}}
	\item \par{\textbf{Semestrální práce:}}
	\item \par{\textbf{Semestrální práce:}}
\end{enumerate}

\end{document}

